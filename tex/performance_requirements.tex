The performance of the robot is going to be purely based on the competition requirements. With the competition consisting of two rounds, this robot must be capable in performing well in both rounds to the best of our abilities. The first round involves underwater navigation through rings to grab a block. A robotic arm must be designed to handle this task. The robot must then traverse back through the rings and place the block on a shelf. Afterwards the robot races to the finish line to tally the points earned for the round.The second round involves unlocking a box underwater by pressing a button. This action releases tennis balls that the robot must collect on the surface. Separate from the robotic arm in round one, a mechanism that can scoop up tennis balls from the surface is required. These tennis balls must then be placed in a box above the water's surface.


This robot must perform all of its schedule tasks within a 30 minute time limit. If the robot does not perform well enough in round one, then the robot will not be eligible for round two. Overall the goal is to complete all the tasks in a given round as soon as possible. The battery supply must be able to sustain charge for 60 minutes total worst case assuming all time is used. The robot must be completed and ready to compete T-20 minutes prior to the start of round one on April 2, 2022.


\subsection{Mobility/Frame}

\subsubsection{Description}

The robot must be able to traverse reliably through a user interface, which will likely work in conjunction with a camera feed. This involves being able to move in three dimensions in a consistent manner that can easily controlled by a pilot. Priorities will be focused primarily on being able to dive vertically and keep a constant depth, then adding horizontal capabilities afterwards. Since this robot needs to navigate through rings and will likely have a tether, the robot must be able to traverse forwards and backwards on command due to the risk of the tether getting caught in the tunnel rings. The mobility system will be based on using multiple thrusters that have algorithms to keep them perpendicular to the bottom of the pool for vertical diving. 

\subsubsection{Source}

2022 IEEE R5 Conference Student Robotics Competition Rules

\subsubsection{Constraints}

The entire robot, assuming that bonus points are going to be obtainable, must be no more than 28 lbs. (and an absolute limit of 30 lbs.) Ideally the robot should weigh less than 15 lbs. to obtain the maximum amount of bonus points. Being lighter is overall a great goal to achieve due to the nature of the robot design. The lighter it is, the less power needed to propel the robot and in theory it should also be easier to control. The size must also remain with a 20" x 20" x 20" cube. The smaller the robot too, the easier the mobility system will be to implement. The robot must also stay within a tight budget. To receive maximum bonus points the total itemized list must be less than \$500. Bonus points are reduced until the budget goes in excess of \$2,000.


The robot must not be powered by AC power, and the robot must not use fuses greater than 30 Amps. A maximum of 48 volts can be used on the robot.

\subsubsection{Priority}

Critical


\subsection{Robotic Arm}

\subsubsection{Description}

This robotic arm must be able to list a trash block that is about 6" cubed in space.

\subsubsection{Source}

2022 IEEE R5 Conference Student Robotics Competition Rules

\subsubsection{Constraints}

The robot arm when extended must not extend past the dimensions of the maximum size allowable robot and must fall within the weight requirements.

\subsubsection{Priority}

High


\subsection{Scooping Mechanism}

\subsubsection{Description}

This robotic arm must be able to list a trash block that is about 6" cubed in space.

\subsubsection{Source}

2022 IEEE R5 Conference Student Robotics Competition Rules

\subsubsection{Constraints}

The scooping mechanism must not extend past the dimensions of the maximum size allowable robot and must fall within the weight requirements.

\subsubsection{Priority}

High