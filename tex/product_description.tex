The objective of the robot is to demonstrate the use of a tethered underwater robot to collect 'trash' from the ocean floor, midwater, and surface. The robot then will deposit the trash in a proper receptacle. The game field simulates an underwater environment containing objects such as industrial infrastructure and underwater debris. Underwater and surface obstacles and anomalies represent typical operational challenges. 

\subsection{Features \& Functions}
The frame of the robot will be made on carbon fiber. Carbon fiber tubes will produce a cuboid, whose ends will be connected by 3D printed parts. The robot will be equipped with 4-5 thrusters (number of thrusters tentative) controlled by microcontroller(s). The robot will also have a camera that will help the pilot to see underwater. A pressure transducer may also be utilized to detect and control depth (This is not confirmed). A claw/grabbing mechanism will allow the robot to grab and hold objects underwater. The robot will have a tether that will house the wires (for connecting microcontrollers to thrusters and claw), Cat 5 cable (to connect camera to computer). The robot will also contain a battery used to power it. 

Off site, we will have our computer for the camera feed and to see other data, and a remote with joystick to control the robot. 

\subsection{External Inputs \& Outputs}
Through the tether: 

The camera will output a live feed to the computer. 

A cable may be used to output data about thruster usage to the computer.  description and use.

Input: 

Through the tether, microcontroller will provide instructions for the thrusters and claw. 

The battery will provide power to the thrusters. 

\subsection{Product Interfaces}
We will have a live camera feed through the Cat 5 cable. 

We will have graphs and other data representing thruster power, usage, and other data. 

We may also have information about depth (if pressure transducer is used). 
