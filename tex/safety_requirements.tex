Include a header paragraph specific to your product here. Safety requirements might address items specific to your product such as: no exposure to toxic chemicals; lack of sharp edges that could harm a user; no breakable glass in the enclosure; no direct eye exposure to infrared/laser beams; packaging/grounding of electrical connections to avoid shock; etc.

\subsection{Laboratory equipment lockout/tagout (LOTO) procedures}
\subsubsection{Description}
Any fabrication equipment provided used in the development of the project shall be used in accordance with OSHA standard LOTO procedures. Locks and tags are installed on all equipment items that present use hazards, and ONLY the course instructor or designated teaching assistants may remove a lock. All locks will be immediately replaced once the equipment is no longer in use.
\subsubsection{Source}
CSE Senior Design laboratory policy
\subsubsection{Constraints}
Equipment usage, due to lock removal policies, will be limited to availability of the course instructor and designed teaching assistants.
\subsubsection{Standards}
Occupational Safety and Health Standards 1910.147 - The control of hazardous energy (lockout/tagout).
\subsubsection{Priority}
Critical

\subsection{National Electric Code (NEC) wiring compliance}
\subsubsection{Description}
Any electrical wiring must be completed in compliance with all requirements specified in the National Electric Code. This includes wire runs, insulation, grounding, enclosures, over-current protection, and all other specifications.
\subsubsection{Source}
CSE Senior Design laboratory policy
\subsubsection{Constraints}
High voltage power sources, as defined in NFPA 70, will be avoided as much as possible in order to minimize potential hazards.
\subsubsection{Standards}
NFPA 70
\subsubsection{Priority}
Critical

\subsection{RIA robotic manipulator safety standards}
\subsubsection{Description}
Robotic manipulators, if used, will either housed in a compliant lockout cell with all required safety interlocks, or certified as a "collaborative" unit from the manufacturer.
\subsubsection{Source}
CSE Senior Design laboratory policy
\subsubsection{Constraints}
Collaborative robotic manipulators will be preferred over non-collaborative units in order to minimize potential hazards. Sourcing and use of any required safety interlock mechanisms will be the responsibility of the engineering team.
\subsubsection{Standards}
ANSI/RIA R15.06-2012 American National Standard for Industrial Robots and Robot Systems, RIA TR15.606-2016 Collaborative Robots
\subsubsection{Priority}
Critical
